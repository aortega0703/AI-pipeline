Since ancient times humankind has made an effort to catalogue the stars visible
in the sky; from the $36$ stars listed in the ``three stars each'' by the
ancient Babylonians in the $2000s\ B.C.$~\cite{astronomy:history:north}, to the
last full-sky catalogue to date with over $1.5$ billion objects, the Gaia Data
Release 3~\cite{gaia:dr3:esa}, made available to the public on June 13, 2022, by
the European Space Agency.

With the advent of modern telescopes and satellites, the capability to detect
even the dimmest lights from space have reached heights previously unthinkable.
The gaia catalogue is but one of the many catalogues available to date, which
serves to illustrate sheer quantity of data currently collected. 

This massive volume of data means that most of it will never get to be studied
by the hand of an astronomer, who has to prioritize human effort in
studying objects that are already known to possess qualities of special interest.
In order to identify these special observations from the sea of data available,
multiple techniques have been developed~\cite{pulsar:dataset:explanation:lyon}.
In the current paper data from a specialized pulsar
catalogue~\cite{pulsar:dataset:lyon} is approached using multiple Artificial
Intelligence techniques in order to identify possible pulsar star candidates.

In Section \ref{sec:problem} the astronomical phenomenom is described more in
depth along with the problem to tackle in this. Section \ref{sec:methods} is a
short description of each of the methods utilized along with the process
previous to their use. Section \ref{sec:results} deals with the most noteworthy
results along with a comparison of the performance over all methods. Finally
Section \ref{sec:conclusion} is a brief discussion of the results obtained, what
they might entail, and future work.