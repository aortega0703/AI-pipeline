
\subsection{Support vector machine}

The SVM algorithm relies on solving an optimization problem, and in the current
implementation this is done using gradient descent. The cost function result
of the use of Lagrange multipliers to solve the problem, can be seen in
Figure~\ref{fig:svn:cost} that is converging.

\begin{figure*}[t]
    \includegraphics[width=\linewidth]{svn:cost.png}
    \caption{Cost function of the optimization problem for SVM
    \label{fig:svn:cost}}
\end{figure*}

\subsection{Neural network}

Neural networks are blackbox models that one may not know with precise certainty
what meta parameters to use for each problem. So in order to explore a small
part of the space and find the best network configuration, multiple nets where
trained. The meta parameters to vary are \textit{number of layers, number of
neurons per layer,} and \textit{learning rate}, all with 2500 epochs of
training. 

Figure \ref{fig:nn} holds then the most note worthy of these networks
being, minimum error, maximum error, minimum gradient and maximum gradient.
In this it is possible to observe that the network with maximum error is so
because it started near a local minima. Also, not necessarily more layers and
neurons equivalates to a better network, \todo[inline]{Describe image}  

\begin{figure*}[ht]
    \begin{subfigure}[t]{0.49\linewidth}
        \includegraphics[width=\linewidth]{nn:err:min.png}
        \caption{Gradient and error for the minimal error NN
        \label{fig:nn:err:min}}
    \end{subfigure}
    \hfill
    \begin{subfigure}[t]{0.49\linewidth}
        \includegraphics[width=\linewidth]{nn:err:max.png}
        \caption{Gradient and error for the maximal error NN
        \label{fig:nn:err:max}}
    \end{subfigure}
    
    \begin{subfigure}[b]{0.49\linewidth}
        \includegraphics[width=\linewidth]{nn:grad:min.png}
        \caption{Gradient and error for the minimal gradient NN
        \label{fig:nn:grad:min}}
    \end{subfigure}
    \hfill
    \begin{subfigure}[b]{0.49\linewidth}
        \includegraphics[width=\linewidth]{nn:grad:max.png}
        \caption{Gradient and error for the maximal gradient NN
        \label{fig:nn:grad:max}}
    \end{subfigure}
    \caption{Gradient and error of the most notable NN obtained 
        \label{fig:nn}}
\end{figure*}

\subsection{Performance}

Unsupervised methods may end up having a different number of classes than those
provided by the dataset. For this reason all comparisons of performance will not
be performed point-wise where each point is marked as right or wrong; but
instead comparing pair of points and checking if they are in the same, or
different classes according to the labels. This being said, Tables
\ref{tab:index:train} through \ref{tab:index:validation} show how each of the
classification methods stack up to a variety of indices. Highest index values
where achieved by the neural network with minimum error followed closely by the
one with minimum gradient. It is possible that the $\phi$ score of mountain
clustering, substractive clustering, K-means and fuzzy C-means is caused because
they have more than $2$ classes and the index measures only the quality of
binary classification.

\begin{table*}
    \csvreader[
        centered tabular=l|ccccccccccc,
        column count=12,
        no head,
        late after first line = {\\\hline},
    ]{../tables/index:train.csv}{}%
    {\csvlinetotablerow}%
    \caption{Multiple indices on training data\label{tab:index:train}}
\end{table*}

\begin{table*}
    \csvreader[
        centered tabular=l|ccccccccccc,
        column count=12,
        no head,
        late after first line = {\\\hline},
    ]{../tables/index:test.csv}{}%
    {\csvlinetotablerow}%
    \caption{Multiple indices on testing data\label{tab:index:test}}
\end{table*}

\begin{table*}
    \csvreader[
        centered tabular=l|ccccccccccc,
        column count=12,
        no head,
        late after first line = {\\\hline},
    ]{../tables/index:validation.csv}{}%
    {\csvlinetotablerow}%
    \caption{Multiple indices on validation data\label{tab:index:validation}}
\end{table*}



\section{Unsupervised leaning\label{sec:unsupervised}}