\documentclass[conference]{IEEEtran}
\IEEEoverridecommandlockouts
% The preceding line is only needed to identify funding in the first footnote.
% If that is unneeded, please comment it out.
\usepackage{cite}
\usepackage{amsmath,amssymb,amsfonts}
\usepackage{algorithm2e}
\usepackage{graphicx}
\graphicspath{ {../images/} }
\usepackage{textcomp}
\usepackage{xcolor}
\usepackage{todonotes}
\def\BibTeX{{\rm B\kern-.05em{\sc i\kern-.025em b}\kern-.08em
    T\kern-.1667em\lower.7ex\hbox{E}\kern-.125emX}}
\begin{document}

\title{Automated Labeling of Pulsar Star Candidates}

\author{\IEEEauthorblockN{Andrés Felipe Ortega Montoya}
\IEEEauthorblockA{\textit{Mathematical Engineering} \\
Medellín, Colombia \\
aortega7@eafit.edu.co}}

\maketitle

\begin{abstract}
\todo{}
\end{abstract}

\begin{IEEEkeywords}
AI, Pulsar, Cluster, Supervised, Unsupervised
\end{IEEEkeywords}

\section{Introduction}
\todo{}
The name Pulsar (from \underline{Puls}ating st\underline{ar}) refers to a
type of neutron star with powerfull magnetic fields.\cite{nasa:pulsar}
This field reaches so great of a force that it accelerates particles that 
ultimately get thrown into its poles generating and ultimately heating them.
These poles act as hotspots that radiate heat into space, much more than
the rest of the star. As the star spins, these beams of electromagnetic
waves (mostly on the X-ray spectrum) go in and out of view from earth at an
extremely regular periods. This motion makes the star appear as pulsating
in the sky, thus, giving it its name.

The number of stars in the \todo{bobeda celeste} is far too great to be
classified by hand, having the \todo{registro de estrellas n} objects as of
date. It has then been increasingly important to design techniques that ease
the process of exploring this data. Among these it is to note the great impact
that AI has had on the area of astronomy, being able to label correctly
data points at an unprecedented pace. \todo{}To this day there is more data
being collected than what could ever be checked by astronomers, so the current
aproach is to search the giant databases for particular objects of interest and
those results are considered candidates for classification until verified by
some agency. 

\section{Sampling}

In order to decide the best criterion for sampling


\begin{figure}[h]
    \includegraphics[width=\linewidth]{sample:dist:train}
    \caption{Slice of the training data distribution. \label{fig:sample:dist:train}}    
\end{figure}

\begin{figure*}[h]
    \includegraphics[width=\linewidth]{sample:hist}
    \caption{Example of a figure caption. \label{fig:sample:hist}}
\end{figure*}

\begin{figure}[h]
    \includegraphics[width=\linewidth]{sample:info}
    \caption{Example of a figure caption. \label{fig:sample:info}}
\end{figure}

\begin{algorithm}
    $P = \{0.6,\ 0.2,\ 0.2\}$\\
    $P' \gets \{\}$\\
    $I \gets \{x:\ 0\leq x\leq \lvert S\rvert,\ x\in\mathbb{N}\}$\\
    \For{$p \in P$}{
        $F(x) \gets \text{PDF} \textbf{ with support } I$\\
        $s \gets p\lvert S\rvert \textbf{ realizations from } F(x)$\\
        $P' \gets P' \cup \{s\}$\\
        $I \gets I - s$\\
    }
    \caption{Sampling algorithm}\label{sampling:alg}
\end{algorithm}
\bibliographystyle{IEEEtran}
\bibliography{references}
\end{document}
