\documentclass[conference]{IEEEtran}
\IEEEoverridecommandlockouts
% The preceding line is only needed to identify funding in the first footnote.
% If that is unneeded, please comment it out.
\usepackage{cite}
\usepackage{amsmath,amssymb,amsfonts}
\usepackage{algorithm2e}
\usepackage{graphicx}
\graphicspath{ {../images/} }
\usepackage{textcomp}
\usepackage{xcolor}
\usepackage{todonotes}
\def\BibTeX{{\rm B\kern-.05em{\sc i\kern-.025em b}\kern-.08em
    T\kern-.1667em\lower.7ex\hbox{E}\kern-.125emX}}
\begin{document}

\title{Automated Labeling of Pulsar Star Candidates}

\author{\IEEEauthorblockN{Andrés Felipe Ortega Montoya}
    \IEEEauthorblockA{\textit{Mathematical Engineering} \\
        Medellín, Colombia \\
        aortega7@eafit.edu.co}}

\maketitle

\begin{abstract}
    \todo{}
\end{abstract}

\begin{IEEEkeywords}
    AI, Pulsar, Labeling
\end{IEEEkeywords}

\section{Introduction\label{sec:intro}}
It is know that since ancient times humankind humankind has made an effort to
catalogue the stars visible in the sky, from the $36$ stars listed in the
\"three stars each\" by the ancient Babylonians in the $2000s\ B.C.$
\cite{astronomy:history:north}; to the last full-sky catalogue to date with over
$1.5$ billion objects, the Gaia Data Release 3\cite{gaia:dr3:esa}, made available
to the public on June 13, 2022, by the European Space Agency.

With the advent of modern telescopes and satellites, the capability to detect
even the dimmest lights from space have reached heights previously unthinkable.
The gaia catalogue is but one of the many catalogues available to date, which
serves to illustrate sheer quantity of data currently collected. Current
telescopes are constantly surveying the sky in search for evermore objects to study.

This massive volume of data means that most of it will never get to be studied
by hand by astronomers, instead better focusing their efforts studying object
that are already known to possess qualities that may result of special interest.
In order to identify these special observations from the sea of data available,
multiple techniques have been developed.\cite{pulsar:dataset:explanation:lyon} In the
current paper, data from a specialized pulsar catalogue\cite{pulsar:dataset:lyon},
is approached using multiple Artificial Intelligence techniques in order to
identify possible pulsar star candidates.
\todo{Sections list and definition}

\section{The Problem\label{sec:problem}}
The name Pulsar (from \underline{Puls}ating st\underline{ar}) refers to rapidly
rotating and strongly magnetized neutron stars.\cite{pulsar:definition:nasa}
The magnetic field present in these stars, accelerates particles to high
speeds around the star, which may ultimately get thrown into the magnetic
poles poles of the pulsar, heating them. This heat gradient gets to be so high
that the poles may act as hotspots, radiating massive amounts of heat into space,
distinctively more than that of the rest of the star. As the star spins,
these beams of electromagnetic waves (mostly on the X-ray spectrum) go in and out
of view from earth at extremely regular periods, motion that makes it appear as
pulsating in the sky, thus, giving it its name.

Pulsars get to occupy an special place in solving a wide range of physical and
astronomical problems.\cite{pulsar:importance:kramer} The extreme conditions on
these allow to test the limits of gravitational theories, solid state physics and
plasma physics under extreme conditions among others. For most uses one needs
not to know how pulsars work but treating them as natural clocks with stabilities
similar to the best atomic clocks over time spans of months or years. 
Correctly identifying these objects presents then a challenge worth approaching.

Although over short periods of time, the signals received from each pulsar
varies slightly on each rotation,\cite{pulsar:importance:kramer} it is possible
to make an appropriate labeling using average measures over longer periods of time.
In practice however, their dim signals means that almost all detections are a
result of radio frequency interference and noise\cite{pulsar:dataset:explanation:lyon}.
creating the necessity of finding new ways to differentiate real candidates
from the rest.

\section{State of the art\label{sec:state_of_art}}

\todo{}

\section{Sampling\label{sec:sampling}}

In order to decide the best criterion for sampling


\begin{figure}[h]
    \includegraphics[width=\linewidth]{sample:dist:train}
    \caption{Slice of the training data distribution. \label{fig:sample:dist:train}}
\end{figure}

\begin{figure*}[h]
    \includegraphics[width=\linewidth]{sample:hist}
    \caption{Example of a figure caption. \label{fig:sample:hist}}
\end{figure*}

\begin{figure}[h]
    \includegraphics[width=\linewidth]{sample:info}
    \caption{Example of a figure caption. \label{fig:sample:info}}
\end{figure}

\begin{algorithm}
    $P = \{0.6,\ 0.2,\ 0.2\}$\\
    $P' \gets \{\}$\\
    $I \gets \{x:\ 0\leq x\leq \lvert S\rvert,\ x\in\mathbb{N}\}$\\
    \For{$p \in P$}{
        $F(x) \gets \text{PDF} \textbf{ with support } I$\\
        $s \gets p\lvert S\rvert \textbf{ realizations from } F(x)$\\
        $P' \gets P' \cup \{s\}$\\
        $I \gets I - s$\\
    }
    \caption{Sampling algorithm}\label{sampling:alg}
\end{algorithm}
\bibliography{references.bib}
\bibliographystyle{IEEEtran}
\end{document}
