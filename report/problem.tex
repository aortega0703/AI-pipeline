The name Pulsar (from \underline{Puls}ating st\underline{ar}) refers to rapidly
rotating and strongly magnetized neutron stars~\cite{pulsar:definition:nasa}.
The magnetic field present in these stars, accelerates particles to high speeds
around the star, which ultimately get thrown into the magnetic poles of the
pulsar, heating them. The heat gradient between the poles and the rest of the
star gets so high that the poles may act as hotspots, radiating massive amounts
of heat into space forming two distinctive beams of electromagnetic waves
(mostly on the X-ray spectrum). As the star spins, these beams go in and out of
view from earth at extremely regular periods, motion that makes it appear as
pulsating in the sky, thus, giving it its name.

Pulsars get to occupy an special place in solving a wide range of physical and
astronomical problems~\cite{pulsar:importance:kramer}. The extreme conditions on
these allow to test the limits of gravitational theories, solid state physics
and plasma physics among others. Pulsars act as natural clocks with stabilities
similar to the best atomic clocks over time-spans of months or years. Correctly
identifying these objects presents then a challenge worth approaching.

Although over short periods of time, the signals received from each pulsar
varies slightly on each rotation~\cite{pulsar:importance:kramer}, it is possible
to make an appropriate labeling of them using average measures over longer
periods of time. In practice however, their dim signals means that almost every
detection is a result of radio frequency interference (RFI) and
noise~\cite{pulsar:dataset:explanation:lyon}; creating the necessity of finding
new ways to differentiate real candidates from the rest.

\citet{pulsar:dataset:lyon} provides a dataset which contains a total of $17898$
data points, $1639$ for them corresponding to real pulsar detections (verified
by human annotators) and, $16259$ detections associated to RFI or noise. The
dataset was modified and split in Kaggle~\cite{pulsar:dataset:kaggle} into one
unlabelled set with $5370$ observations and a labelled set with $12528$
observations, the latter being the one used on this paper for purposes of
training, testing, and validation of the techniques used.

As previously mentioned, the detection of pulsars necessitates to collect a few
hundred to thousand of pulses together in order to discern them from the
background noise, this collection is \textit{folded} into an integrated pulse
profile~\cite{pulsar:importance:kramer}. The dataset has 8 continuous variables per
pulsar candidate, each with its corresponding binary labeling. The first $4$ of
them are statistics that describe the integrated pulse profile of the star,
while the last $4$ are similarly statistics from the DM-SNR (Dispersion Measure
- Signal to Noise Ratio)~\cite{pulsar:dataset:explanation:lyon}. The task being
then to decide based on these 8 variables if any given object is a possible
pulsar or not.

In the order that they appear on the database, the variables are:

\begin{table}[ht]
    \begin{tabular}{l|l}
        $X_0:$ & Mean of the integrated profile.\\
        $X_1:$ & Standard deviation of the integrated profile.\\
        $X_2:$ & Excess kurtosis of the integrated profile.\\
        $X_3:$ & Skewness of the integrated profile.\\
        $X_4:$ & Mean of the DM-SNR curve.\\
        $X_5:$ & Standard deviation of the DM-SNR curve.\\
        $X_6:$ & Excess kurtosis of the DM-SNR curve.\\
        $X_7:$ & Skewness of the DM-SNR curve.\\
        $Y:$   & Class 
    \end{tabular}
    \caption{Attributes describing a pulsar candidate\label{tab:variables}}
\end{table}